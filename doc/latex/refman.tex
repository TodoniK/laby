\documentclass[twoside]{book}

% Packages required by doxygen
\usepackage{calc}
\usepackage{doxygen}
\usepackage{graphicx}
\usepackage[utf8]{inputenc}
\usepackage{makeidx}
\usepackage{multicol}
\usepackage{multirow}
\usepackage{textcomp}
\usepackage[table]{xcolor}

% NLS support packages
\usepackage[french]{babel}

% Font selection
\usepackage[T1]{fontenc}
\usepackage{mathptmx}
\usepackage[scaled=.90]{helvet}
\usepackage{courier}
\usepackage{amssymb}
\usepackage{sectsty}
\renewcommand{\familydefault}{\sfdefault}
\allsectionsfont{%
  \fontseries{bc}\selectfont%
  \color{darkgray}%
}
\renewcommand{\DoxyLabelFont}{%
  \fontseries{bc}\selectfont%
  \color{darkgray}%
}

% Page & text layout
\usepackage{geometry}
\geometry{%
  a4paper,%
  top=2.5cm,%
  bottom=2.5cm,%
  left=2.5cm,%
  right=2.5cm%
}
\tolerance=750
\hfuzz=15pt
\hbadness=750
\setlength{\emergencystretch}{15pt}
\setlength{\parindent}{0cm}
\setlength{\parskip}{0.2cm}
\makeatletter
\renewcommand{\paragraph}{%
  \@startsection{paragraph}{4}{0ex}{-1.0ex}{1.0ex}{%
    \normalfont\normalsize\bfseries\SS@parafont%
  }%
}
\renewcommand{\subparagraph}{%
  \@startsection{subparagraph}{5}{0ex}{-1.0ex}{1.0ex}{%
    \normalfont\normalsize\bfseries\SS@subparafont%
  }%
}
\makeatother

% Headers & footers
\usepackage{fancyhdr}
\pagestyle{fancyplain}
\fancyhead[LE]{\fancyplain{}{\bfseries\thepage}}
\fancyhead[CE]{\fancyplain{}{}}
\fancyhead[RE]{\fancyplain{}{\bfseries\leftmark}}
\fancyhead[LO]{\fancyplain{}{\bfseries\rightmark}}
\fancyhead[CO]{\fancyplain{}{}}
\fancyhead[RO]{\fancyplain{}{\bfseries\thepage}}
\fancyfoot[LE]{\fancyplain{}{}}
\fancyfoot[CE]{\fancyplain{}{}}
\fancyfoot[RE]{\fancyplain{}{\bfseries\scriptsize Généré le Mardi Avril 11 2023 15\-:04\-:34 pour Labyrinthe en C par Doxygen }}
\fancyfoot[LO]{\fancyplain{}{\bfseries\scriptsize Généré le Mardi Avril 11 2023 15\-:04\-:34 pour Labyrinthe en C par Doxygen }}
\fancyfoot[CO]{\fancyplain{}{}}
\fancyfoot[RO]{\fancyplain{}{}}
\renewcommand{\footrulewidth}{0.4pt}
\renewcommand{\chaptermark}[1]{%
  \markboth{#1}{}%
}
\renewcommand{\sectionmark}[1]{%
  \markright{\thesection\ #1}%
}

% Indices & bibliography
\usepackage{natbib}
\usepackage[titles]{tocloft}
\setcounter{tocdepth}{3}
\setcounter{secnumdepth}{5}
\makeindex

% Hyperlinks (required, but should be loaded last)
\usepackage{ifpdf}
\ifpdf
  \usepackage[pdftex,pagebackref=true]{hyperref}
\else
  \usepackage[ps2pdf,pagebackref=true]{hyperref}
\fi
\hypersetup{%
  colorlinks=true,%
  linkcolor=blue,%
  citecolor=blue,%
  unicode%
}

% Custom commands
\newcommand{\clearemptydoublepage}{%
  \newpage{\pagestyle{empty}\cleardoublepage}%
}


%===== C O N T E N T S =====

\begin{document}

% Titlepage & ToC
\hypersetup{pageanchor=false}
\pagenumbering{roman}
\begin{titlepage}
\vspace*{7cm}
\begin{center}%
{\Large Labyrinthe en C }\\
\vspace*{1cm}
{\large Généré par Doxygen 1.8.5}\\
\vspace*{0.5cm}
{\small Mardi Avril 11 2023 15:04:34}\\
\end{center}
\end{titlepage}
\clearemptydoublepage
\tableofcontents
\clearemptydoublepage
\pagenumbering{arabic}
\hypersetup{pageanchor=true}

%--- Begin generated contents ---
\chapter{Index des structures de données}
\section{Structures de données}
Liste des structures de données avec une brève description \-:\begin{DoxyCompactList}
\item\contentsline{section}{\hyperlink{structLabyrinthe}{Labyrinthe} \\*Structure du labyrinthe }{\pageref{structLabyrinthe}}{}
\end{DoxyCompactList}

\chapter{Index des fichiers}
\section{Liste des fichiers}
Liste de tous les fichiers avec une brève description \-:\begin{DoxyCompactList}
\item\contentsline{section}{include/\hyperlink{labyrinthe_8h}{labyrinthe.\-h} \\*Ce fichier contient les prototypes des fonctions du labyrinthe ainsi que les commentaires associés à ces fonctions }{\pageref{labyrinthe_8h}}{}
\end{DoxyCompactList}

\chapter{Documentation des structures de données}
\hypertarget{structLabyrinthe}{\section{Référence de la structure Labyrinthe}
\label{structLabyrinthe}\index{Labyrinthe@{Labyrinthe}}
}


Structure du labyrinthe.  




{\ttfamily \#include $<$labyrinthe.\-h$>$}

\subsection*{Champs de données}
\begin{DoxyCompactItemize}
\item 
char $\ast$ \hyperlink{structLabyrinthe_a5a95fe6fa4f565d397c586f3b5988ab1}{grille}
\item 
int \hyperlink{structLabyrinthe_ab7faa6700dd69e633b1d6c1c8d1974a1}{hauteur}
\item 
int \hyperlink{structLabyrinthe_a23007607a374817790805e23ed1be08c}{largeur}
\item 
int \hyperlink{structLabyrinthe_a58b3b04b825e29eaa7b3e8750261dcd1}{x\-Entree}
\item 
int \hyperlink{structLabyrinthe_ad004da5f699104c8df81a4a88fc8e53a}{y\-Entree}
\item 
int \hyperlink{structLabyrinthe_ac74b66a45b9e340fcb9857701316d348}{genere}
\item 
int \hyperlink{structLabyrinthe_ad27777cb14b150736ce76ce2182b2968}{resolu}
\end{DoxyCompactItemize}


\subsection{Description détaillée}
Structure du labyrinthe. 

\subsection{Documentation des champs}
\hypertarget{structLabyrinthe_ac74b66a45b9e340fcb9857701316d348}{\index{Labyrinthe@{Labyrinthe}!genere@{genere}}
\index{genere@{genere}!Labyrinthe@{Labyrinthe}}
\subsubsection[{genere}]{\setlength{\rightskip}{0pt plus 5cm}int Labyrinthe\-::genere}}\label{structLabyrinthe_ac74b66a45b9e340fcb9857701316d348}
\hypertarget{structLabyrinthe_a5a95fe6fa4f565d397c586f3b5988ab1}{\index{Labyrinthe@{Labyrinthe}!grille@{grille}}
\index{grille@{grille}!Labyrinthe@{Labyrinthe}}
\subsubsection[{grille}]{\setlength{\rightskip}{0pt plus 5cm}char$\ast$ Labyrinthe\-::grille}}\label{structLabyrinthe_a5a95fe6fa4f565d397c586f3b5988ab1}
\hypertarget{structLabyrinthe_ab7faa6700dd69e633b1d6c1c8d1974a1}{\index{Labyrinthe@{Labyrinthe}!hauteur@{hauteur}}
\index{hauteur@{hauteur}!Labyrinthe@{Labyrinthe}}
\subsubsection[{hauteur}]{\setlength{\rightskip}{0pt plus 5cm}int Labyrinthe\-::hauteur}}\label{structLabyrinthe_ab7faa6700dd69e633b1d6c1c8d1974a1}
\hypertarget{structLabyrinthe_a23007607a374817790805e23ed1be08c}{\index{Labyrinthe@{Labyrinthe}!largeur@{largeur}}
\index{largeur@{largeur}!Labyrinthe@{Labyrinthe}}
\subsubsection[{largeur}]{\setlength{\rightskip}{0pt plus 5cm}int Labyrinthe\-::largeur}}\label{structLabyrinthe_a23007607a374817790805e23ed1be08c}
\hypertarget{structLabyrinthe_ad27777cb14b150736ce76ce2182b2968}{\index{Labyrinthe@{Labyrinthe}!resolu@{resolu}}
\index{resolu@{resolu}!Labyrinthe@{Labyrinthe}}
\subsubsection[{resolu}]{\setlength{\rightskip}{0pt plus 5cm}int Labyrinthe\-::resolu}}\label{structLabyrinthe_ad27777cb14b150736ce76ce2182b2968}
\hypertarget{structLabyrinthe_a58b3b04b825e29eaa7b3e8750261dcd1}{\index{Labyrinthe@{Labyrinthe}!x\-Entree@{x\-Entree}}
\index{x\-Entree@{x\-Entree}!Labyrinthe@{Labyrinthe}}
\subsubsection[{x\-Entree}]{\setlength{\rightskip}{0pt plus 5cm}int Labyrinthe\-::x\-Entree}}\label{structLabyrinthe_a58b3b04b825e29eaa7b3e8750261dcd1}
\hypertarget{structLabyrinthe_ad004da5f699104c8df81a4a88fc8e53a}{\index{Labyrinthe@{Labyrinthe}!y\-Entree@{y\-Entree}}
\index{y\-Entree@{y\-Entree}!Labyrinthe@{Labyrinthe}}
\subsubsection[{y\-Entree}]{\setlength{\rightskip}{0pt plus 5cm}int Labyrinthe\-::y\-Entree}}\label{structLabyrinthe_ad004da5f699104c8df81a4a88fc8e53a}


La documentation de cette structure a été générée à partir du fichier suivant \-:\begin{DoxyCompactItemize}
\item 
include/\hyperlink{labyrinthe_8h}{labyrinthe.\-h}\end{DoxyCompactItemize}

\chapter{Documentation des fichiers}
\hypertarget{labyrinthe_8h}{\section{Référence du fichier include/labyrinthe.h}
\label{labyrinthe_8h}\index{include/labyrinthe.\-h@{include/labyrinthe.\-h}}
}


Ce fichier contient les prototypes des fonctions du labyrinthe ainsi que les commentaires associés à ces fonctions.  


{\ttfamily \#include $<$stdio.\-h$>$}\\*
{\ttfamily \#include $<$stdlib.\-h$>$}\\*
\subsection*{Structures de données}
\begin{DoxyCompactItemize}
\item 
struct \hyperlink{structLabyrinthe}{Labyrinthe}
\begin{DoxyCompactList}\small\item\em Structure du labyrinthe. \end{DoxyCompactList}\end{DoxyCompactItemize}
\subsection*{Énumérations}
\begin{DoxyCompactItemize}
\item 
enum \hyperlink{labyrinthe_8h_a80bfcb579205425260198f4d062009ac}{enum\-Caractere} \{ \\*
\hyperlink{labyrinthe_8h_a80bfcb579205425260198f4d062009aca435fc7d5b53120363c3c3a0506b11588}{C\-O\-U\-L\-O\-I\-R}, 
\hyperlink{labyrinthe_8h_a80bfcb579205425260198f4d062009aca4812eae6f742a8f18779779eeeffd570}{M\-U\-R}, 
\hyperlink{labyrinthe_8h_a80bfcb579205425260198f4d062009acaa1f529db9f2d334d9c41dcc5f6f6e2a7}{N\-O\-R\-D}, 
\hyperlink{labyrinthe_8h_a80bfcb579205425260198f4d062009acad2d3ceab3fce63bbe07882ab5a4e991e}{S\-U\-D}, 
\\*
\hyperlink{labyrinthe_8h_a80bfcb579205425260198f4d062009acab26f94a551dbfca6e0f4f6b5ad145c01}{E\-S\-T}, 
\hyperlink{labyrinthe_8h_a80bfcb579205425260198f4d062009aca564dddee8796eda21f35c76cb586aeb6}{O\-U\-E\-S\-T}, 
\hyperlink{labyrinthe_8h_a80bfcb579205425260198f4d062009acad7e097bda6d981de2520f49fe74c25b7}{M\-A\-X}
 \}
\begin{DoxyCompactList}\small\item\em Structure de case possible. \end{DoxyCompactList}\end{DoxyCompactItemize}
\subsection*{Fonctions}
\begin{DoxyCompactItemize}
\item 
int \hyperlink{labyrinthe_8h_af9ddf46939443c5cf807f5d575ee4afa}{get\-Valeur\-Case} (\hyperlink{structLabyrinthe}{Labyrinthe} lab, int x, int y)
\begin{DoxyCompactList}\small\item\em Fonction permettant de retourner la valeur située dans la case en x,y. \end{DoxyCompactList}\item 
void \hyperlink{labyrinthe_8h_a2deaef054f87dcd623632c48c762d20a}{set\-Valeur\-Case} (\hyperlink{structLabyrinthe}{Labyrinthe} lab, int x, int y, int nouvelle\-Valeur)
\begin{DoxyCompactList}\small\item\em Fonction permettant de mettre à jour la valeur située dans la case en x,y. \end{DoxyCompactList}\item 
void \hyperlink{labyrinthe_8h_a696e4b25519feade2d645cb639609ef3}{afficher\-Laby} (\hyperlink{structLabyrinthe}{Labyrinthe} lab)
\begin{DoxyCompactList}\small\item\em Fonction permettant d'afficher le labyrinthe. \end{DoxyCompactList}\item 
void \hyperlink{labyrinthe_8h_ab67ba509442dbb415efd428a7eeb76dd}{creuser\-Laby} (\hyperlink{structLabyrinthe}{Labyrinthe} lab, int x, int y)
\begin{DoxyCompactList}\small\item\em Fonction implémentant l'algorithme de reverse backtracking. \end{DoxyCompactList}\item 
void \hyperlink{labyrinthe_8h_af2d815e5d1e73b3bc426e69cb929b9a3}{generer\-Laby} (\hyperlink{structLabyrinthe}{Labyrinthe} lab)
\begin{DoxyCompactList}\small\item\em Fonction permettant d'initialiser le labyrinthe à un ensemble de murs, et de creuser les cases de celui-\/ci en appelant la fonction creuser\-Laby pour toutes ses cases. \end{DoxyCompactList}\item 
int \hyperlink{labyrinthe_8h_a75007e98e4b82e6d4673a93a345a2eb9}{get\-Direction\-Opposee} (int direction\-Actuelle)
\begin{DoxyCompactList}\small\item\em Fonction permettant de placer la fleche vers l'entrée. \end{DoxyCompactList}\end{DoxyCompactItemize}


\subsection{Description détaillée}
Ce fichier contient les prototypes des fonctions du labyrinthe ainsi que les commentaires associés à ces fonctions. 

\subsection{Documentation du type de l'énumération}
\hypertarget{labyrinthe_8h_a80bfcb579205425260198f4d062009ac}{\index{labyrinthe.\-h@{labyrinthe.\-h}!enum\-Caractere@{enum\-Caractere}}
\index{enum\-Caractere@{enum\-Caractere}!labyrinthe.h@{labyrinthe.\-h}}
\subsubsection[{enum\-Caractere}]{\setlength{\rightskip}{0pt plus 5cm}enum {\bf enum\-Caractere}}}\label{labyrinthe_8h_a80bfcb579205425260198f4d062009ac}


Structure de case possible. 

\begin{Desc}
\item[Valeurs énumérées]\par
\begin{description}
\index{C\-O\-U\-L\-O\-I\-R@{C\-O\-U\-L\-O\-I\-R}!labyrinthe.\-h@{labyrinthe.\-h}}\index{labyrinthe.\-h@{labyrinthe.\-h}!C\-O\-U\-L\-O\-I\-R@{C\-O\-U\-L\-O\-I\-R}}\item[{\em 
\hypertarget{labyrinthe_8h_a80bfcb579205425260198f4d062009aca435fc7d5b53120363c3c3a0506b11588}{C\-O\-U\-L\-O\-I\-R}\label{labyrinthe_8h_a80bfcb579205425260198f4d062009aca435fc7d5b53120363c3c3a0506b11588}
}]\index{M\-U\-R@{M\-U\-R}!labyrinthe.\-h@{labyrinthe.\-h}}\index{labyrinthe.\-h@{labyrinthe.\-h}!M\-U\-R@{M\-U\-R}}\item[{\em 
\hypertarget{labyrinthe_8h_a80bfcb579205425260198f4d062009aca4812eae6f742a8f18779779eeeffd570}{M\-U\-R}\label{labyrinthe_8h_a80bfcb579205425260198f4d062009aca4812eae6f742a8f18779779eeeffd570}
}]\index{N\-O\-R\-D@{N\-O\-R\-D}!labyrinthe.\-h@{labyrinthe.\-h}}\index{labyrinthe.\-h@{labyrinthe.\-h}!N\-O\-R\-D@{N\-O\-R\-D}}\item[{\em 
\hypertarget{labyrinthe_8h_a80bfcb579205425260198f4d062009acaa1f529db9f2d334d9c41dcc5f6f6e2a7}{N\-O\-R\-D}\label{labyrinthe_8h_a80bfcb579205425260198f4d062009acaa1f529db9f2d334d9c41dcc5f6f6e2a7}
}]\index{S\-U\-D@{S\-U\-D}!labyrinthe.\-h@{labyrinthe.\-h}}\index{labyrinthe.\-h@{labyrinthe.\-h}!S\-U\-D@{S\-U\-D}}\item[{\em 
\hypertarget{labyrinthe_8h_a80bfcb579205425260198f4d062009acad2d3ceab3fce63bbe07882ab5a4e991e}{S\-U\-D}\label{labyrinthe_8h_a80bfcb579205425260198f4d062009acad2d3ceab3fce63bbe07882ab5a4e991e}
}]\index{E\-S\-T@{E\-S\-T}!labyrinthe.\-h@{labyrinthe.\-h}}\index{labyrinthe.\-h@{labyrinthe.\-h}!E\-S\-T@{E\-S\-T}}\item[{\em 
\hypertarget{labyrinthe_8h_a80bfcb579205425260198f4d062009acab26f94a551dbfca6e0f4f6b5ad145c01}{E\-S\-T}\label{labyrinthe_8h_a80bfcb579205425260198f4d062009acab26f94a551dbfca6e0f4f6b5ad145c01}
}]\index{O\-U\-E\-S\-T@{O\-U\-E\-S\-T}!labyrinthe.\-h@{labyrinthe.\-h}}\index{labyrinthe.\-h@{labyrinthe.\-h}!O\-U\-E\-S\-T@{O\-U\-E\-S\-T}}\item[{\em 
\hypertarget{labyrinthe_8h_a80bfcb579205425260198f4d062009aca564dddee8796eda21f35c76cb586aeb6}{O\-U\-E\-S\-T}\label{labyrinthe_8h_a80bfcb579205425260198f4d062009aca564dddee8796eda21f35c76cb586aeb6}
}]\index{M\-A\-X@{M\-A\-X}!labyrinthe.\-h@{labyrinthe.\-h}}\index{labyrinthe.\-h@{labyrinthe.\-h}!M\-A\-X@{M\-A\-X}}\item[{\em 
\hypertarget{labyrinthe_8h_a80bfcb579205425260198f4d062009acad7e097bda6d981de2520f49fe74c25b7}{M\-A\-X}\label{labyrinthe_8h_a80bfcb579205425260198f4d062009acad7e097bda6d981de2520f49fe74c25b7}
}]\end{description}
\end{Desc}


\subsection{Documentation des fonctions}
\hypertarget{labyrinthe_8h_a696e4b25519feade2d645cb639609ef3}{\index{labyrinthe.\-h@{labyrinthe.\-h}!afficher\-Laby@{afficher\-Laby}}
\index{afficher\-Laby@{afficher\-Laby}!labyrinthe.h@{labyrinthe.\-h}}
\subsubsection[{afficher\-Laby}]{\setlength{\rightskip}{0pt plus 5cm}void afficher\-Laby (
\begin{DoxyParamCaption}
\item[{{\bf Labyrinthe}}]{lab}
\end{DoxyParamCaption}
)}}\label{labyrinthe_8h_a696e4b25519feade2d645cb639609ef3}


Fonction permettant d'afficher le labyrinthe. 


\begin{DoxyParams}[1]{Paramètres}
\mbox{\tt in}  & {\em lab} & \-: pointeur sur la structure du labyrinthe \\
\hline
\end{DoxyParams}
\hypertarget{labyrinthe_8h_ab67ba509442dbb415efd428a7eeb76dd}{\index{labyrinthe.\-h@{labyrinthe.\-h}!creuser\-Laby@{creuser\-Laby}}
\index{creuser\-Laby@{creuser\-Laby}!labyrinthe.h@{labyrinthe.\-h}}
\subsubsection[{creuser\-Laby}]{\setlength{\rightskip}{0pt plus 5cm}void creuser\-Laby (
\begin{DoxyParamCaption}
\item[{{\bf Labyrinthe}}]{lab, }
\item[{int}]{x, }
\item[{int}]{y}
\end{DoxyParamCaption}
)}}\label{labyrinthe_8h_ab67ba509442dbb415efd428a7eeb76dd}


Fonction implémentant l'algorithme de reverse backtracking. 


\begin{DoxyParams}[1]{Paramètres}
\mbox{\tt in}  & {\em lab} & \-: pointeur sur la structure du labyrinthe \\
\hline
\mbox{\tt in}  & {\em x} & \-: abscisse de la case dont on veut visiter (et creuser) les voisins \\
\hline
\mbox{\tt in}  & {\em y} & \-: ordonnée de la case dont on veut visiter (et creuser) les voisins \\
\hline
\end{DoxyParams}
\hypertarget{labyrinthe_8h_af2d815e5d1e73b3bc426e69cb929b9a3}{\index{labyrinthe.\-h@{labyrinthe.\-h}!generer\-Laby@{generer\-Laby}}
\index{generer\-Laby@{generer\-Laby}!labyrinthe.h@{labyrinthe.\-h}}
\subsubsection[{generer\-Laby}]{\setlength{\rightskip}{0pt plus 5cm}void generer\-Laby (
\begin{DoxyParamCaption}
\item[{{\bf Labyrinthe}}]{lab}
\end{DoxyParamCaption}
)}}\label{labyrinthe_8h_af2d815e5d1e73b3bc426e69cb929b9a3}


Fonction permettant d'initialiser le labyrinthe à un ensemble de murs, et de creuser les cases de celui-\/ci en appelant la fonction creuser\-Laby pour toutes ses cases. 


\begin{DoxyParams}[1]{Paramètres}
\mbox{\tt in}  & {\em lab} & \-: pointeur sur la structure du labyrinthe \\
\hline
\end{DoxyParams}
\hypertarget{labyrinthe_8h_a75007e98e4b82e6d4673a93a345a2eb9}{\index{labyrinthe.\-h@{labyrinthe.\-h}!get\-Direction\-Opposee@{get\-Direction\-Opposee}}
\index{get\-Direction\-Opposee@{get\-Direction\-Opposee}!labyrinthe.h@{labyrinthe.\-h}}
\subsubsection[{get\-Direction\-Opposee}]{\setlength{\rightskip}{0pt plus 5cm}int get\-Direction\-Opposee (
\begin{DoxyParamCaption}
\item[{int}]{direction\-Actuelle}
\end{DoxyParamCaption}
)}}\label{labyrinthe_8h_a75007e98e4b82e6d4673a93a345a2eb9}


Fonction permettant de placer la fleche vers l'entrée. 


\begin{DoxyParams}[1]{Paramètres}
\mbox{\tt in}  & {\em direction\-Actuelle} & \-: direction actuelle dans laquelle on creuse \\
\hline
\end{DoxyParams}
\hypertarget{labyrinthe_8h_af9ddf46939443c5cf807f5d575ee4afa}{\index{labyrinthe.\-h@{labyrinthe.\-h}!get\-Valeur\-Case@{get\-Valeur\-Case}}
\index{get\-Valeur\-Case@{get\-Valeur\-Case}!labyrinthe.h@{labyrinthe.\-h}}
\subsubsection[{get\-Valeur\-Case}]{\setlength{\rightskip}{0pt plus 5cm}int get\-Valeur\-Case (
\begin{DoxyParamCaption}
\item[{{\bf Labyrinthe}}]{lab, }
\item[{int}]{x, }
\item[{int}]{y}
\end{DoxyParamCaption}
)}}\label{labyrinthe_8h_af9ddf46939443c5cf807f5d575ee4afa}


Fonction permettant de retourner la valeur située dans la case en x,y. 


\begin{DoxyParams}[1]{Paramètres}
\mbox{\tt in}  & {\em x} & \-: abscisse de la case souhaitée \\
\hline
\mbox{\tt in}  & {\em y} & \-: ordonnée de la case souhaitée \\
\hline
\end{DoxyParams}
\hypertarget{labyrinthe_8h_a2deaef054f87dcd623632c48c762d20a}{\index{labyrinthe.\-h@{labyrinthe.\-h}!set\-Valeur\-Case@{set\-Valeur\-Case}}
\index{set\-Valeur\-Case@{set\-Valeur\-Case}!labyrinthe.h@{labyrinthe.\-h}}
\subsubsection[{set\-Valeur\-Case}]{\setlength{\rightskip}{0pt plus 5cm}void set\-Valeur\-Case (
\begin{DoxyParamCaption}
\item[{{\bf Labyrinthe}}]{lab, }
\item[{int}]{x, }
\item[{int}]{y, }
\item[{int}]{nouvelle\-Valeur}
\end{DoxyParamCaption}
)}}\label{labyrinthe_8h_a2deaef054f87dcd623632c48c762d20a}


Fonction permettant de mettre à jour la valeur située dans la case en x,y. 


\begin{DoxyParams}[1]{Paramètres}
\mbox{\tt in}  & {\em x} & \-: abscisse de la case souhaitée \\
\hline
\mbox{\tt in}  & {\em y} & \-: ordonnée de la case souhaitée \\
\hline
\end{DoxyParams}

%--- End generated contents ---

% Index
\newpage
\phantomsection
\addcontentsline{toc}{part}{Index}
\printindex

\end{document}
